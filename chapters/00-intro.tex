\chapter{Introduction}\label{chap:introduction}
\largerpage
The organization of the lexicon, and especially the relations between groups of lexemes, is a strongly debated topic in linguistics. Some authors have insisted on the lack of any structure in the lexicon.
In this vein, \textcite[3]{DiSciullo.1987} claim that ``[t]he lexicon is like a prison -- it contains only the lawless, and the only thing that its inmates have in common is lawlessness.''
In the alternative view, the lexicon is assumed to have a rich structure that captures all regularities and partial regularities that exist between lexical entries.

Two very different schools of linguistics have insisted on the organization of the lexicon.
On the one hand, for theories like \textsc{hpsg} (Head-driven Phrase Structure Grammar) \autocite{Pollard.1994}, but also some versions of construction grammar \autocite{Fillmore.1995}, the lexicon is assumed to have a very rich structure which captures common grammatical properties between its members.
In this approach, a type hierarchy organizes the lexicon according to common properties between items.
For example \textcite[4, among others]{Koenig.1999}, working from an \textsc{hpsg} perspective, claims that the lexicon ``provides a unified model for partial regularties, medium-size generalizations, and truly productive processes.''

On the other hand, from the perspective of \isi{usage-based} linguistics, several authors have drawn attention to the fact that lexemes which share morphological or syntactic properties tend to be organized in clusters of surface (phonological or semantic) similarity \autocites{Bybee.1982, Eddington.1996, Skousen.1989}.
This approach, often called analogical, has developed highly accurate computational and non-computational models that can predict the classes to which lexemes belong.
Like the organization of lexemes in type hierarchies, analogical relations between items help speakers to make sense of intricate systems and reduce apparent complexity \autocite{Kopcke.1984}.

\largerpage
Despite this core commonality, and despite the fact that most linguists seem to agree that analogy plays an important role in language, there has been remarkably little work on bringing together these two approaches.
Formal grammar traditions have been very successful in capturing grammatical behaviour but, in the process, have downplayed the role analogy plays in linguistics \autocite{Anderson.2015}.
In this work, I aim to change this state of affairs.
First, by providing an explicit formalization of how analogy interacts with grammar, and second, by showing that analogical effects and relations closely mirror the structures in the lexicon.
I will show that both formal grammar approaches and usage-based analogical models capture mutually compatible relations in the lexicon.

This book is divided into two parts. Part I consists of two chapters. Chapter \ref{chap:problems} presents a summary of the most relevant work on analogy and delimits the exact kind of analogy I will focus on in the rest of the book. Because of its longstanding tradition in linguistics, there are various definitions and uses of analogy, not all of which are relevant to the present investigation. Chapter \ref{chap:solution} presents the basic tools for integrating analogy into grammar and introduces the main system and its predictions. This chapter contains the main theoretical claim put forward in this book, namely that analogy is intrinsically linked to type hierarchies in the lexicon.

Part II is divided into six chapters, containing nine case studies.
Chapter \ref{chap:method} introduces the neural networks used for modelling analogy and discusses the basic tools for evaluating model performance (kappa scores and accuracy).
Chapter \ref{chap:gender-assignment} presents two case studies on the gender-inflection class interaction in Latin and Romanian. In these examples I show how the correlations and discrepancies between gender and inflection class in nouns can be modelled using multiple inheritance hierarchies, and how the shapes of these hierarchies are clearly reflected in the analogical relations.
Chapter \ref{chap:hybrid} discusses the effects of hybrid types in morphological phenomena in Russian and Croatian. These two languages present cases where for a single morphological property, the grammar offers two mutually exclusive, competing alternatives. In Russian, I show an example from derivational doubletism in the diminutive system, and in Croatian I present an overabundance example from the instrumental singular.
Chapter \ref{chap:structural} explores systems where the morphological process clearly has an effect on the features analogy operates on.
The use of prefixes for inflection in Swahili and Otomi cause the analogical relations to take place mostly at the beginning of the stems.
In Hausa, due to the use of broken plurals, the analogical models require a much more structural representation.
Finally, Chapter \ref{chap:complex} deals with two systems that show high complexity and a large number of inflection classes: Spanish verb inflection, and Kasem plural and singular markers.
In both Spanish and Kasem, the inflection class system requires multiple inflectional dimensions that operate independently from each other, but interact to produce the inflection classes of verbs (Spanish) and nouns (Kasem).
In both of these examples we see clear reflexes of the multiple dimensions of inflection in the analogical relations.

The two most important chapters are Chapters \ref{chap:solution} and \ref{chap:complex}. The chapters in Part II stand on their own and are mostly self contained. The empirical results reported in these chapters stand independently of the theory of this book. 
